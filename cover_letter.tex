\documentclass[parskip]{scrartcl}
\usepackage[utf8]{inputenc}
\usepackage[english]{babel}
\usepackage[document]{ragged2e}
 
\begin{document}
 
\pagestyle{empty}

\begin{flushright}
    \Large
    %25th of July 2018
    \today
\end{flushright}

\begin{center}
    \huge
    Cover Letter\\
    \vspace{0.4cm}
    \LARGE
    A wearable fall detection system based \\ on Body
    Area Networks\\
    \vspace{0.4cm}
    \large
    Luigi La Blunda, Lorena Gutiérrez-Madroñal, Matthias F. Wagner, Inmaculada Medina-Bulo
\end{center}
 
    \vspace{0.8cm}

    \normalsize
Dear Editor,

all authors have checked this manuscript and have agreed to submit it to IEEE Transactions on Mobile Computing, to be considered for publication. This manuscript is the authors' original work and has not been published, nor has been submitted simultaneously elsewhere.

The main contributions of this paper are as follows:
A portable fall detection system based on a Body Area Network (BAN) is presented. The solution provided enables a precise fall detection due to its unique hardware architecture and an additional classification of fall types.
Based on acceleration data we are able to detect fall events.
The BAN generates a large amount of real time data that must be processed with low latency in order to provide fast assistance in the case of a fall. For this reason, the Complex Event Processing (CEP) technology is used, which enables a fast evaluation of data. Using CEP, event patterns and rules are created to detect safety critical situations, in our case, the fall events.

People can fall in many different ways and it is very important that our system can reliably detect them. In order to analyze the reliability of our system and to determine vulnerabilities, it is important to generate test events that simulate different types of falls. This task will be solved using the IoT-TEG tool, which is able to automatically generate test events for each type of fall. In addition, a fusion of acceleration and electrocardiogram (ECG) data was introduced, which leads to a more accurate fall detection by adding health-related information. Subsequently, a part of the complete System-Theoretic Accident Model and Processes (STAMP) hazard analysis was applied to the system, allowing detailed analysis of possible hazards that could lead to the malfunction of the system. The STAMP analysis will be applied to all levels of the system within the ongoing prototype development process. \newpage

We are convinced that our portable fall detection solution provides essential assistance to the population in case of life-threatening situations.


We appreciate the opportunity to submit our manuscript and for considering the review of our manuscript. We highly appreciate your time and look forward to your response.

Kind regards,

Luigi La Blunda, Lorena Gutiérrez-Madroñal,  Matthias F. Wagner, and Inmaculada Medina Bulo



\end{document}



