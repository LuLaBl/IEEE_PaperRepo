\documentclass[parskip]{scrartcl}
\usepackage[utf8]{inputenc}
\usepackage[english]{babel}
\usepackage[document]{ragged2e}
 
\begin{document}
 
\pagestyle{empty}

\begin{flushright}
    \Large
    %25th of July 2018
    \today
\end{flushright}

\begin{center}
    \huge
    Reuse of previously published work \\
    \vspace{0.4cm}

\end{center}
 
    \vspace{0.8cm}

    \normalsize

Dear Editor,

according to the guideline in \textbf{Subsection 8.2.10}, on \textbf{page 122} of the \textbf{PSPB Operations Manual}, we would like to point out that this publication contains parts of a previously published paper \cite{LorenaFall}: \textit{Test event generation for a fall-detection IoT system}, in the IEEE Internet of Things Journal. The above work \cite{LorenaFall} serves as a basis for the current paper, which contains a further development of our fall detection system, but also a more profound research of the problem. In other words, this paper represents the improved prototype from the prior paper \cite{LorenaFall} with new features that increase the efficiency and reliability of our prototype.

As we have mentioned in this paper we are talking about an ongoing project. We have published our first proposal in \cite{LorenaFall}. So, if we compare each section of this new paper to \cite{LorenaFall}:
\begin{itemize}
	\item \textbf{Section 1:} this section differs cause in this paper we are using  different reasons/references to point out the fall problem. Only minor arguments are identical to \cite{LorenaFall}, (80-90\% is new content).
	\item \textbf{Section 2:} in this paper we are more focused on available fall detection solutions, \cite{LorenaFall} points out more the event generators, only a minor part is about fall detection where 2 solutions were explained, (approximately 80 \% is new).
	\item \textbf{Section 3:} is quite identical cause this is the physical behavior of the fall. In this paper, we have added the new method to split the fall into 4 phases, which allows you to analyze/detect it in a more precise way, (10-20\% is new). \textbf{Subsections 3.2 and 3.3} are identical to \cite{LorenaFall} because we are describing the tool and basic definitions that are necessary to understand the whole project. 
	\item \textbf{Section 4:} in comparison to \cite{LorenaFall} we do not take the first prototype into consideration, here we are presenting a second prototype, which was briefly introduced as an improved prototype in \cite{LorenaFall}. Additionally, we have been more detailed in the description of the system and we have added a new aspect: the three axis reference scheme to show why the belt is able to provide a more accurate fall detection. So, Subsection 4.1. contains 70\% new content.
	\begin{itemize}
		\item \textbf{Subsections 4.2.1 and 4.2.2}, are used to define the fall event which is generated with IoT-TEG, this part is very similar to \cite{LorenaFall}, but it contains a shorter description because we are focused in other relevant parts of our work. 
		\item \textbf{Subsection 4.3:} introduces the sensor fusion, it is completely new since we are talking about the fusion of medical and physical sensors. Moreover, we added information about the ECG tests that have been conducted. Additionally, we are talking about the results of the tests performed in the medical lab and the introduction of the new developed ECG harness. The master's thesis of our student based on ECG was included as well. 
		\item  \textbf{Subsection 4.4:} only some points are identical to \cite{LorenaFall}. Here we added the problem related to baseline wander which is comprised in the ECG signal and we updated the problem for the fall analysis, (approximately 60-70\% is new).
	\end{itemize}
	\item \textbf{Section 5:} this is completely new because in this section we are applying the improved prototype, (100\% is new).
	\item \textbf{Section 6:} it is new as well since we are talking about the research questions which we introduced in the introduction, (100\% new content).
	\item \textbf{Section 7:} when the conclusions of this paper and \cite{LorenaFall} are compared it can be seen that there is a connection, given that it is an ongoing project. In this paper, we are introducing an improvement of the previous work which was stated in \cite{LorenaFall}. Especially because we stated the application of STAMP in \cite{LorenaFall} and in this paper we are applying it, (80 - 90\% new content).
	
	So, in general, we can assume that 70-80\% of the paper is new. The repeated information is because we are working with an ongoing project and we are using some of the same tools for testing. Moreover, the main tool is being improved and we are introducing the new changes that we mentioned in \cite{LorenaFall}. Finally, the background is practically the same because, as far as we know, nowadays there are no new papers that are working in this specific area.
	
	
	
	
\end{itemize}

%----------------------------------------References--------------------------------------
\bibliographystyle{IEEEtran}
\bibliography{references}
\end{document}



