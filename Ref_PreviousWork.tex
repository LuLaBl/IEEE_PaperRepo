\documentclass[parskip]{scrartcl}
\usepackage[utf8]{inputenc}
\usepackage[english]{babel}
\usepackage[document]{ragged2e}
 
\begin{document}
 
\pagestyle{empty}

\begin{flushright}
    \Large
    %25th of July 2018
    \today
\end{flushright}

\begin{center}
    \huge
    Reuse of previously published work \\
    \vspace{0.4cm}

\end{center}
 
    \vspace{0.8cm}

    \normalsize

Dear Editor,

according to the guideline in \textbf{Subsection 8.2.10}, on page 122 of the \textbf{PSPB Operations Manual}, we would like to point out that this publication contains parts of a previously published paper: \textit{Test event generation for a fall-detection IoT system}, in the IEEE Internet of Things Journal. The above work serves as a basis for the current paper, which contains a further development of our fall detection system, but also a more profound research of the problem. In other words, this paper represents the improved prototype from the prior paper with new features that increase the efficiency and reliability of our prototype.

A more detailed research of the problem is presented by analyzing several fall detection approaches, which are described in the chapter "Related work". These comparisons contributed to improving our fall detection prototype of the previously submitted paper. 

In the chapter "Background" the fall event was divided into 4 phases: \textit{prefall phase, falling phase, impact phase and postfall phase}. Since a fall is caused by daily activities, this division allows a more efficient fall analysis in contrast to the above-mentioned method described in the paper.
In addition, a more detailed analysis was presented which motivates the use of CEP. 

The fourth chapter illustrates the hardware architecture of the prototype, which emphasizes and clarifies the advantages of this developed architecture using the 3-axis reference scheme (Figure 6). In addition, the integration of the ECG signal is explained, which allows a comfortable use due to the newly developed elastic harness. Tests with medical consultation provided valuable information that can be used for further development. 

An important addition to this paper is the STAMP hazard analysis, which is essential for the development of safety-critical systems, by analyzing all system levels and determining the cause of the system malfunction. \cite{FallRepo}

%----------------------------------------References--------------------------------------
\bibliographystyle{IEEEtran}
\bibliography{references}
\end{document}



